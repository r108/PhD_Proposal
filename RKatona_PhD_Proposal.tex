% $Id: AllegProposal.tex,v 1.8 2000/07/05 21:02:12 culver Exp $
% AllegProposal.tex
% by A. Thall
% 13. Feb 2003
%
% Small edits and a few additions made by R. Roos
% 21 Jan 2007
% Most particularly, the "box" around the thesis statement has been removed,
% section titles have been modified. The section named "Prior work II" has
% been commented out. The \topmargin has been changed to -.5in and the
% change to \parindent has been commented out.
% The filename "nausicaa.eps" has been changed to simply "nausicaa" so that
% pdflatex can be used on the file (and a file named "nausicaa.pdf" has
% been created using the "epstopdf" command).
% Several subsections have been added to illustrate subsection usage.
% The word "comp" has been replaced by "project" or "thesis" throughout.
% Other small changes have been made.
%
% This document provides a sample Senior Project Proposal template for use
% by students in Allegheny's CS and Applied Computing programs.

%\NeedsTeXFormat{LaTeX2e}
\documentclass[11pt]{article}

%The following is used by WinEdt to set up cross-referencing to the BibTeX files
%It is NOT commented out---the comment lets it be simply ignored by non-WinEdt LaTeX compilers

%GATHER{mybibtexDB.bib}
\usepackage[utf8]{inputenc}
\usepackage{setspace}
\usepackage{amsmath}
\usepackage{amssymb}
\usepackage{epsfig}
\usepackage{fancybox}
\usepackage{listings}
%\usepackage{algo}
\usepackage{url}
%\usepackage{cleveref}
\usepackage[numbers,sort&compress]{natbib}
%\usepackage{hyperref}
\usepackage[hidelinks]{hyperref}
\usepackage{verbatim}
\usepackage{fancyhdr}
%\setlength{\headheight}{55.2pt}
\setlength{\textheight}{9in}
\setlength{\textwidth}{6in}
\setlength{\oddsidemargin}{.25in}
\setlength{\topmargin}{-.5in}  % changed from -.25 by RSR on 1/21/07
%\parindent .5in    % commented out by RSR 1/21/07
\fancyheadoffset{.2in}
\pagestyle{fancy}
\fancyhf{}
\lhead{}
\chead{}
\rhead{}
\cfoot{\thepage}

%\usepackage[utf8]{inputenc}
\usepackage{color}
\usepackage{xcolor}
\hypersetup{
	colorlinks,
	linkcolor={gray!50!black},
	citecolor={blue!50!black},
	urlcolor={blue!80!black}
}

%put words in the hyphenation statement if you want to enforce
%how LaTeX should break them (or not) at the end of a line.
%\hyphenation{repre-sen-tations problems exact linear}
\hyphenation{itself}

%%%%%
%% Commented out -- RSR, 1/21/07
%%%%%
% The following provides a box to surround the thesis statement
%\newenvironment{Thesis}%
%{\begin{Sbox}\begin{minipage}{.95\linewidth}}%
%{\end{minipage}\end{Sbox}\begin{center}\fbox{\TheSbox}\end{center}}

\title{PhD Research Proposal}
\author{by \\ \\ Roland Katona \\ \\ Supervisors:  Dr. Donna O'Shea and  Dr. Dirk Pesch}

\begin{document}

% You can specify a language and other options for
% the code-formatting "listings" package
%\lstset{language=C++,basicstyle=\small,
 %       stringstyle=\ttfamily,showstringspaces=false}

\singlespace
\maketitle



\begin{abstract}   
%\normalsize
The explosion of technological innovations in the past decades has irreversibly changed
our way of life. Cities are evolving and becoming smarter and more interactive as new
technologies are added to the existing infrastructure. Small embedded devices with sensing and acting capabilities are contributing to the creation of an additional sensing layer to our
habitats with the aim of making them easier and safer to live and work in. Wireless Sensor Networks (WSNs) play a pivotal role in this by providing valuable information to simplify the management and improve the utilization of scarce urban resources.

WSNs, since their inception, were typically designed and
deployed as a vertically integrated homogeneous sensing infrastructure developed with a specific application or user in mind. Such systems often suffer from issues such as redundant deployment of similar sensing capabilities within the same sensing boundary resulting in issues such as: low return of investment and resource underutilization. 
Nonetheless, future WSNs in Smart Cities should be designed in a way where multiple users can exploit the same infrastructure. Heterogeneity of devices/platforms, standards and technologies, however, encumbers the extensive realization of shared WSNs. Therefore, to accomplish such a shared infrastructure, the ownership and management of the sensor network needs to be separated.  Additionally, traditional WSN development methods impose numerous fundamental problems. These include difficulties in application programming for the highly resource constrained and heterogeneous nodes/platforms as well as limitations on (re)configure-ability of the nodes/applications and the lack of interoperability amongst the different WSN networks/platforms. 



To address these issues and limitations, this PhD research proposes a framework that combines multiple virtualisation methods at the different layers (hardware, system, network, application/service) of the WSN infrastructure. Virtualisation of WSNs creates opportunities where resources and services from several WSNs can be efficiently used by multiple applications and users, bringing with it a host of new challenges and research opportunities. Applying virtualisation to the WSN is expected to simplify network management, accelerate application development and enhance security by abstracting and isolating the underlying resources. Virtualisation can also enable concurrency and facilitate interoperability amongst heterogeneous nodes/platforms by providing a generic execution environment. 
Therefore the ultimate goal of the proposed framework is to create a dynamic, adaptable and scalable virtualised WSN infrastructure that meets the business objectives of network providers by enabling them to quickly provision and deliver value-added services to various customers. 

\begin{comment}

Moreover, managing such an environment requires a policy based management approach to drive  The scope of the PhD proposal outlined within this document aims to investigate how WSNs can benefit from combining various virtualisation techniques. 

Currently, wireless sensor networks (WSN) tend to be owned by a single organization
and programmed for a specific function, application or service. Besides, the installation
and deployment of each WSN is managed independently by their owner. This traditional
service model is often referred to as a ’walled garden’. Within this ’walled garden’ prob-
lems such as resource and network under-utilization arising from redundant deployment
of similar sensor networks at the same location can emerge. This can lead to a poor
return of investment from a sensor network. Furthermore, such set-up poses additional
limitations on the (re)programmability of heterogeneous nodes and lacks interoperability
of different networks. Hence, the inefficiency of the traditional approach to WSN design,
which advocates a single application that serves a single user, is inherent.
Opening of the aforementioned ’walled garden’ however is now possible with the ad-
vancements in Cloud computing and virtualization technology. By means of virtualiza-
tion, the traditional benefits of separating the network and its ownership would also be
applied such as: economies of scale; reduced cost of ownership and decreased cost to
customers. Moreover, the abstraction of heterogeneous hardware elements allows and
facilitates the (re)programmability of wireless sensor networks, and enables the sharing
and aggregation of networking and node resources via standard interfaces. Generic ab-
straction layers that hide the discrepancy of the underlying hardware are expected to
foster the basis for simplifying development of new as well as existing WSN deployments.
Hence, such a model has the potential to support various services and applications which
can exploit arbitrary resources that may be owned by different authorities. Additionally,
the integration of several sensor networks into a federated architecture would enable in-
creased networking capabilities and higher degree of interoperability of the different WSN
deployments that in turn would provide greater choice of service offerings and improved
cost efficiency.
With the adoption of SDN and Virtualization paradigms into an urban WSN environ-
ment, the described research attempts to address the above outlined limitations of current
WSN implementations and to overcome the design barriers of traditional WSN deploy-
ments. The realization of these paradigms in wireless sensor networks, however, is known
to be a rather challenging task. It is therefore the purpose of this PhD work to investigate
the viability of enabling software defined virtualized wireless sensor networks, on top of
existing physical WSNs, which can accommodate multiple concurrent applications and
users, beyond the context, for which the network was originally designed. The project
aims to architect a network model and developing the corresponding implementation
framework that will enable a flexible and adaptable WSN infrastructure, consisting of
heterogeneous nodes, that is easy to manage and maintain through well-defined business
policies.
content...
\end{comment}

\end{abstract}
\setcounter{secnumdepth}{3}
\setcounter{tocdepth}{2}
\doublespace
\pagebreak
\tableofcontents

% This sets section-numbering to only include Section and Subsection numbers


% The original template's content was copied here
%\include{template_content}

% % % % % % % % % % % % % % % % % % % % % % % % % % % % % % % % % % % % % % % % % % %
\pagebreak
\section{Research context and contribution to the research field}

This PhD research is being carried out as part of SFI project SURF: Service-centric networking for urban-scale feedback systems which aims to exploit information
and communications technologies to improve the quality of life of city dwellers around
the world. Urban environments have limited resources such as road networks, energy and
water. These resources are under increasing strain as a result of population growth. Such
resources could be managed in a better way if there was better access to real-time, city-
wide information on how the resources are being used.

The SURF project will investigate the design of communications technologies which improve the accessibility of this information for urban services providers. This will allow improved management of city resources, resulting in less traffic, a reduction in water shortages, and fewer power-cuts.

In addition to the traditional communication networks, evolving technologies such as mobile phones, IP/wireless cameras and wireless sensor nodes are added to the existing infrastructure. And as a result, contributing to the creation of an additional sensing layer to human habitats.

In the aforementioned context, this PhD work will concentrate on developing methods that enable existing and new WSN infrastructures to accommodate multiple users and applications concurrently. The proposed research will investigate various virtualisation techniques that can potentially complement each. The resulting framework implements various virtualisation mechanisms to form a virtual ecosystem aiming to address and improve issues associated with the existing technologies. These issues include ease of (re)programmability,  manageability, security, performance, resource and constraint management as well as interoperability, deployment/maintenance cost and complexity.


%\input{context}
\pagebreak
\section{Objectives}

\begin{itemize}
	\item Analyse how various virtualisation techniques at the different levels of the WSN infrastructure complement each other and examine the synergy that occurs when combining them.
	
	\item Study the impact of virtualisation on the WSN node resources with regards to the potential gain in utilization and the inherent overhead associated with the technique.
	
	\item Investigate the benefits of virtualising a sensor network in relation to service provisioning and network management.
	
	\item Design and develop a virtualisation ecosystem framework that meets the requirements outlined in section 2.1.
	
	\item Evaluate the proposed framework by implementing it within the scope of the SURF project.	
\end{itemize}
\bigskip
\subsection{Identified characteristics of the proposed framework}
This section identifies and analyses the desirable characteristics of the proposed framework.

	\subsubsection{ \textbf{\emph{High level programming abstractions.}}} Traditionally, application development for WSNs involves working close to the operating system and solving low level problems. This makes development more difficult and slower. Bridging the gap between the high level application requirements and the low level hardware operations necessitates the availability of sufficient high level programming abstractions.
	
	\subsubsection{ \textbf{\emph{Ease of (re)configure-ability.}}} WSN nodes after their deployment tend to be difficult to access therefore it is a non-trivial and in many cases a non-viable task to reconfigure them. Hence an efficient mechanism that advocates over the air (OTA) reprogramming is needed. Furthermore, the energy cost of radio communication is high and the available bandwidth is low. Therefore a compact data format for optimized packet transmission has to be defined.
	
	\subsubsection{ \textbf{\emph{(Re)programmability.}}} The ability to easily implement new user defined functions after deployment -without reloading the entire firmware- is essential in order to respond to changes in the environment or in user requirements. To address this problem, a modular, high level programming scheme is required.   
	
	\subsubsection{ \textbf{\emph{Concurrency.}}} Support for multi-user and multi-application runtime environment, that accommodates the concurrent execution of multiple application instances is a key requirement to maximize resource utilization.
	
	\subsubsection{ \textbf{\emph{Abstraction of heterogeneity.}}} Abstracting the low-level hardware specific details of the heterogeneous WSN nodes (devices with various capabilities potentially from different vendors) and representing them as generic resources is of key importance to assist interoperability and simplifying application development. Additionally, aggregation of resources into a virtual pool offers higher availability and adds extra redundancy to the WSN. 
	
	\subsubsection{\textbf{\emph{Application/user isolation.}}} Multiple applications/users concurrently sharing the same resources should not be aware of each other and perceive it as if they were the sole user of the system. Moreover, isolation should provide protection against data corruption by restricting access to other user's/application's resources.
	
	\subsubsection{ \textbf{\emph{Local dynamic adaptability to environment.}}} The nodes should not be confined to a set of pre-defined functionality nor should they include application specific functionality. (e.g. support for additional functionality through extendible modules)
	
	\subsubsection{ \textbf{\emph{Distributed adaptability.}}} {\color{red} NEEDS TO BE DEFINED!!} The dynamically changing conditions in a WSN necessitate a distributed decision making method which enables applications that are running on individual nodes to adapt to these changes. The ability of the nodes to find optimal solutions collaboratively is an essential ability of future WSNs.
	
	\subsubsection{ \textbf{\emph{Application resource management (reservation and guarantees).}}} {\color{red}MAYBE IT COULD BE MERGED WITH THE BELOW POINT.}
	
	\subsubsection{ \textbf{\emph{Constraint management (resource efficiency).}} }Due to race conditions and risk of over provisioning, proper management and synchronization mechanisms are required for appropriate resource administration. Continuous usage monitoring should be used to allow on-demand up/down scaling of allocated resources.
	
	\subsubsection{ \textbf{\emph{Mobility and migration (application/VM/service).}}} Support for transferring an entire application/VM/service is vital to provide QoS (dynamically changing demand) and redundancy (node/link failure). (e.g. support for autonomous and centralized code propagation over the network.



\pagebreak
\section{Review of the state of the art / current literature / current practice of research}
\bigskip
\subsection{Introduction}

\begin{comment}
The study of Wireless Sensor Networks (WSNs) has increasingly attracted the interest of many researchers in recent years. Their wide range of possible applications have been demonstrated in many areas. These include human habitat, \cite{mainwaring2002wireless}, health \cite{virone2006advanced} and environmental monitoring \cite{4379685,cardell2004field} as well as industry control \cite{1631480} and military surveillance \cite{6268958}.  Advances in the ever-developing field of Micro-Electro-Mechanical Systems (MEMS) in parallel with the latest developments in software technologies have created additional
opportunities to integrate WSNs into our everyday life.  As the field of WSNs matures, ,draws growing attention from industry and academy alike.
%
%\begin{comment}
The explosion of technological innovations in the past decades has irreversibly changed our way of life. Advances in the ever-developing field of Micro-Electro-Mechanical Systems (MEMS), in parallel with the latest developments in software technologies, have created new opportunities to assist our life. By adding computational intelligence to everyday objects they become smarter and more interactive. Together with other embedded devices, such as mobile phones and security cameras, Wireless Sensor Networks (WSNs) contributed to the creation of an additional sensing / actuating layer to our surroundings. This fusion of the various embedded devices into people's living environments is motivated by potential benefits such as simplified management and improved utilization of scarce resources.  

\end{comment}

In recent years, the study of WSNs has attracted the interest of many researchers from industry and academia alike.  The wide range of practical applications of the technology have been demonstrated in many areas. These include human habitat, \cite{mainwaring2002wireless}, health \cite{virone2006advanced} and environmental monitoring \cite{4379685,cardell2004field} as well as industry control \cite{1631480} and military surveillance \cite{6268958}. 


Traditionally, WSNs consist of numerous homogeneous devices, that are programmed for some very specific purpose or application \cite{ammari2013art,mittal2012contemporary}, and also owned and controlled by a single organization \cite{Yick:2008:WSN:1389582.1389832,obaidat2014principles}. WSN nodes can typically be characterized by their physical traits such as power consumption, communication range, memory and storage capacity, computational capability, etc.
Additionally, based on their software functions, the nodes perform two different types of operations. One is node-level operations that involve computational tasks and hardware control such as the radio, RAM, CPU and on-board sensors/actuators. The other one is network-level operations which include communication related tasks such as routing. A more detailed and concise
classification of wireless sensor nodes is provided in the following publications: \cite{davis2012survey,mottola2011programming,Akyildiz2002393}.  

Investigators in the field have recognized that the traditional WSN deployment model inherently lacks flexibility and does not facilitate efficiency in resource utilization \cite{islam2012virtualization}. In order to address these shortcomings many different research efforts were carried out  \cite{virtual_sensor_networks,Leontiadis:2012:STS:2187181.2187188,Fok:2009:AMA:1552297.1552299}. The common objective of these various approaches was to develop more efficient techniques for creating and managing highly customizable and adaptable WSNs which can accommodate multiple concurrent applications running on heterogeneous nodes.  

Virtualisation has been proposed by the WSN research community as a promising solution to address WSN design issues \cite{islam2012survey} and achieve concurrency, improved efficiency and simplified network and service management.
Numerous previous studies have shown how WSNs can benefit from the advantages associated with virtualisation \cite{6076694} including better resource utilization and scalability, ease of service provisioning and migration as well as improved security by isolation. Moreover, virtualisation offers additional benefits through the abstraction of the underlying hardware and processor architectures. By hiding the heterogeneity of device resource and infrastructure details, and by providing high level programming interfaces to programmers, application development can be simplified and accelerated. Consequently, considering the highly distributed quality of WSNs, a generic virtual system representation may bring further benefits, allowing applications to run transparently over disparate physical processor architectures.  

Many viable virtualisation solutions have gained recognition in the general networking \cite{6272301} and computing \cite{6008687} domains. More generally the concept of virtualisation is being applied across virtually all sections of a typical IT infrastructure \cite{murphy2008virtualization}. However, the resource constrained nature of WSNs poses unique challenges that are normally not present in other domains. One of WSNs key characteristic is that the sensor nodes operate from mainly non-renewable battery power. Therefore, extensive research efforts have focused on energy-efficient virtualisation mechanisms to extend battery life, and in turn, prolong operation lifetime \cite{Hong:2012:TEE:2388871.2388872}.  While power consumption is one of the major constraints, there are a number of other limiting factors that are related to WSNs and must be considered when developing a new application.  These limiting factors typically include low memory and computational capability as well as the lack of a simple way to maintain and reprogram sensor nodes after deployment. 

Over the past years several works have been conducted as an attempt to overcome these difficulties and provide effective virtualisation alternatives to traditional WSN deployments. This section presents a summary of the various approaches to virtualisation in WSNs. The predominant focus of this review is to survey existing research works that have demonstrated how WSNs can leverage from the different virtualisation techniques that have been adopted recently. 


 
\subsection{Virtualisation}
The notion of virtualisation has been around for decades, since the debut of virtual memory \cite{denning1970virtual} and virtual computer systems \cite{goldberg1973architectural}.
This section provides a brief introduction to virtualisation and discusses the basic characteristics of the technology.

The concept that provides one or more functional environments, which are independent from the underlying resources, is known as virtualisation. At its core, virtualisation simply refers to the decoupling of computation — processing, sensing/actuating, storage and networking — from the resource that performs the action. It is used to conceal the actual physical traits of computing resources in order to simplify the methods through which those resources can be interacted with. The technology exists in many different forms, depending on the degree of abstraction it provides at the hardware, instruction set, system, process, library or application level. 

%Depending on the desired goal the virtualisation process may divide a single computing resource into multiple virtual copies to advocate concurrency or may combine several independent resource units into one virtual entity to increase computational capacity and supply redundancy. 

One form of virtualisation facilitates multiple application instances to concurrently execute in a virtual environment. Another form allows the aggregation of hardware resources that are physically distributed across multiple devices into an individual virtual unit. The benefits of this is that it offers load balancing, redundancy and abstraction of heterogeneous resources. Such abstractions facilitate the translation of resources into a generic virtual resource that in turn can be consumed by various user applications via high-level software interfaces. 


Virtualisation is typically implemented by applying various techniques such as: hardware/software partitioning, aggregation or composition, time-sharing and indirection or encapsulation \cite{fortes2005guest} which are further outlined below.

\paragraph{Partitioning:} Facilitates the sharing of a single physical resource amongst multiple users or application instances by dividing it into several logical portions (i.e. hard-partitioning). Through multiplexing, many concurrent applications can consume an individual resource by taking turns to access it(i.e. time-sharing).   

\paragraph{Aggregation:} A technique that makes multiple, typically identical, physical units appear to function as a unified logical unit. Thus, the resulting resource pool can be viewed as a single virtual element.

\paragraph{Composition:} A mechanism to combine a number of simpler, typically different, resources into a more complex virtual resource that in turn offers additional functions not provided by any of the original components. 

%Resource allocation and hardware arbitration are the main enablers of virtualisation, through which the substrate physical computing assets %are delineated as software abstractions.  Software isolation for fault mitigation and performance management are also fundamental properties %of virtualisation.

\begin{comment}

\paragraph{Application virtualisation (AV)} is achieved by encapsulating (a.k.a sandboxing) the individual instances of an application into a container from the system runtime environment that they are executed on. Relevant system files and programming libraries are also captured in this container, making applications easily portable across different platforms \cite{lunsford2009virtualization}. 

\paragraph{Network virtualisation (NV)} enables various heterogeneous network  models to co-exist on the same physical infrastructure. NV is defined by the decoupling of the
roles of the traditional Internet Service Providers (ISPs) into
two independent entities [2], [3]: infrastructure providers, who
manage the physical infrastructure, and service providers, who
create virtual networks by aggregating resources from multiple infrastructure providers and offer end-to-end services 
Although the term is rather ambiguous as many different implementations of it can be found, they all share the common objective of logically representing network resources in software. The benefits that are provided by NV include improved manageability and security as well as increased flexibility \cite{Chowdhury2009Network}.

\paragraph{Service virtualisation (SV)} Manifesting itself in various forms, service virtualisation utilises the above definitions. By integrating all necessary elements, it delivers services to user applications over the network. 

Service virtualization is used to simulate the behavior of components in an application so you can perform an accurate and timely test in a world of complex interrelated applications. Production services that may not be available for integration testing can be virtualized so the testing can take place at an appropriate time in the software development process

content...
\end{comment}

\bigskip
Virtualisation technology has now matured to the point where it can be applied to WSNs. By abstracting the hardware specific details of WSN nodes with different architectures and hiding the actual physical environment behind a generic virtual representation, heterogeneous device resources can be automatically mapped to user applications. Virtualisation can offer numerous benefits, by providing user applications with a logical view of the system, including:

\begin{itemize}
	\item Simplified application development
	\item Improved resource utilization
	\item Simplified network management
	\item Increased fault tolerance and enhanced security through isolation
	\item Greater flexibility, in the shape of application compatibility and portability
	
\end{itemize}
 


\subsection{WSN programming}
Today's WSN platforms have greatly benefited from the advances in manufacturing technologies.  Sensor nodes are becoming smaller, cheaper, more efficient and capable in terms of their processing power. Despite the substantial improvements in hardware there are still significant limitations associated with WSN application development. The literature has identified a significant gap \cite{Fok:2009:AMA:1552297.1552299,Levis:2002:MTV:635506.605407,Costa:2007:VMA:1304610.1306527,1621014} which exists between the complexity of the required applications and the lack of sufficient hardware abstractions that would simplify the programming and management of heterogeneous sensor nodes. 

Typically, the Operating System (OS) has the central role of providing Application Programming Interfaces (API) for user applications to the underlying hardware resources \cite{hailperin2007operating,Stallings:1998:OSI:272982}. Memory and CPU management, low-level resource abstraction and task scheduling for multitasking are some of the main responsibilities of a typical OS. 
Due to the unique characteristics of the motes, OS design approaches for WSNs greatly differ from traditional OSs. 

Over the years, important developments have been made to provide light-weight operating systems for WSNs that meet the requirements of diverse and demanding user applications. \cite{1367266,1630599,bhatti2005mantis,eswaran2005nano,s110605900}.A classification framework, which analyses the different approaches that are used for WSN OS design, is presented in \cite{Reddy:2009:WSN:1593545.1593549}. The survey classifies WSN OSs based on their architecture, execution model, reprogramming ability, scheduling and power management. In today's WSN deployments application programming involves working close to the OS and solving low-level system problems that requires technical knowledge which is rarely found amongst application programmers. Hence, the need for suitable high-level programming interfaces is apparent and was identified by \cite{mottola2011programming}. The work classifies the various programming approaches that have gained popularity in WSNs, provides a comprehensive comparison amongst the different schemes and highlights the most important design concerns to program WSN nodes. 

The survey in \cite{mottola2011programming} outlines that the traditional WSNs programming practice is to  create programs that are executed on top of a WSN OS such as Contiki \cite{1367266} or TinyOS \cite{1630599}, hence exploiting the OS' features (e.g. hardware abstractions, multi-threading, etc). This approach, however, infers that the application has to be bound with the OS into a single firmware image and uploaded to the nodes manually and imposes additional limitations that are related to the particular OS' architecture design (Event driven, Multi-threaded, etc.).

Simplifying application development is arguably the main advantage of accessing hardware resources via high level APIs. In WSNs these are normally provided in a number of different ways. Most commonly an OS is used to hide operational complexities from user application logic. Apart from OSs, however, other approaches such as virtualisation with the use of Virtual Machines (VM) \cite{Levis:2002:MTV:635506.605407,simon2005squawk,hong2009tinyvm,4300022} and other middleware \cite{Fok:2009:AMA:1552297.1552299,1621014,6529470,6671886} have been widely proposed.


\subsection{Virtual machines (VMs)}
The benefits offered by a VM execution environment include application portability, multi application support and improved hardware utilization. On one hand, applications running in a VM tend to be slower than their native counterparts due to the overhead associated with byte code interpretation. On the other hand, in many cases the overhead is minimal. Nonetheless, benefits of VMs as flexible computational platforms are well known in mainstream computing\cite{Montero2011750} and in WSN domains alike \cite{Brouwers:2009:DFV:1644038.1644056}.  

Existing VM approaches for WSNs can greatly vary in how they are implemented. For example Squawk \cite{simon2005squawk}, which is a micro edition Java Virtual Machine (JVM) is implemented on the bare metal, while Maté \cite{Levis:2002:MTV:635506.605407}, a stack based byte code interpreter is implemented as a single component of TinyOS \cite{1630599} (an operating system for WSN nodes).

Squawk's core, unlike most VMs, was mainly written in Java that makes it easily portable but currently it is only available for the SunSPOT platform. It implements an application isolation mechanism to allow multiple application instances running concurrently and provides a high level API that enable developers to write wireless applications for WSN in Java.
 
Maté, on the other hand, runs as a single component of TinyOS.  It specifies 24 very compact -one byte long- instructions, 8 of which can be customized for a particular application. In order to reprogram existing WSN deployments, Maté introduces program updates via self-propagating code capsules that are transmitted over the whole network. More complex programs can easily be composed of multiple packets. 
This research work was one of the first successful attempts to implement a VM on a highly resource constrained WSN node. The developers of Maté identified the need for re-tasking as environmental conditions change and borrowed ideas from Active Networks (AN) architecture in which the execution codes are transmitted via active packets. Concurrency at instruction granularity is achieved through three interleaving execution contexts.  Data communication amongst contexts is made possible by a shared variable, a one word heap. Application code running in the VM can be multiplexed with other system tasks, however Maté does not explicitly support multiple concurrent application instances.

The original Maté research was further extended into a framework that enables application developers to generate tailored application specific virtual machines (ASVM) \cite{levis2004bridging} for WSNs. The VMs created by this framework also included an improved code propagation module which was based on the Trickle algorithm \cite{Levis:2004:TSA:1251175.1251177}. 

In addition to the ASVM generator other VM generators were developed over the years. One of which was the VMSTAR framework \cite{koshy2005vmstar} that allows programmers to generate VMs for sensor networks. Apart from the particular application, it also uses hardware specific details as its foundation to optimize its interpreter to the target platform. Extending application functionality is supported in VMSTAR through incremental linking.

Another VM generator is presented in \cite{Palmer:2004:VMG:1267242.1267243}. The research proposes a framework to create customized stack based VMs based on the hardware capabilities of the devices. For resource constrained sensor nodes only a scaled down version is implemented and the full VM for the more powerful ones where the different versions are fully compatible. This approach aims to assist interoperability amongst heterogeneous devices while exploiting the available performance of the devices with regards to their capabilities.  

The research behind Maté was a pioneering work that inspired other researchers and provided a foundation for several VM implementations. The most notable ones include DAViM \cite{Michiels:2006:DDA:1176866.1176868}  and Melete \cite{Yu:2006:SCA:1182807.1182822}.
 
DAViM allows the network to be reprogrammed dynamically by injecting application specific code, and provides isolated execution environments for several running applications.  

Melete adopted Maté's TinyScript, a relatively high level programming language, but also introduced a number of modifications to the original Maté VM. These included enhanced code propagation and application deployment mechanisms and most importantly support for concurrent applications.

Another unique VM approach, MagnetOS, is presented in \cite{Barr:2002:NSS:509526.509528}. It is a distributed OS, implemented as a single system image of a unified Java virtual machine that is installed across the sensor network nodes.

A number of other Java based VMs have also been proposed. TinyVM \cite{hong2009tinyvm} for example -runs atop TinyOS- is a byte code interpreter that implements a subset of the full JVM. Other implementations include Darjeeling \cite{Brouwers:2009:DFV:1644038.1644056}  and TakaTuka \cite{Aslam:2008:ITJ:1460412.1460472}.  

TakaTuka supports all but two of the Java byte-code instructions, threading, synchronized method calls and is fully compliant with the CLDC library \cite{debbabi2006security}, while Darjeeling prioritizes efficiency over functionality and does not provide support for floating point calculations, class reflection or synchronized method calls. 
Application development using Java is a very attractive option due to its wide portability and simplicity to program in a high level object oriented language.

The authors in \cite{895380} describe Scylla, a VM architecture for embedded systems, which differs in its execution model from all of the previously discussed VM implementations. Scylla VMs use a register based execution model as opposed to the stack based model that is used by most embedded VM implementations. Additionally, unlike most VMs Scylla does not interpret byte code, instead Scylla implements an on-the-fly compiler. Hence compiling byte code to native architectures only requires the re-encoding of the instruction words as they can be directly mapped. Furthermore, the register based execution model does not have the overhead associated with the additional PUSH and POP stack operations and also provide optimization by caching the results of previous calculations allowing faster execution of the instructions. On the other hand, stack based execution models can produce more compact codes because long operand addresses do not need to be explicitly specified instead short stack pointers are used. This property makes stack based VMs more attractive for WSNs due to their lower memory requirement and more efficient dissemination of compact byte code. 
\subsection{Virtual Sensor (VS)}
VSs are software entities that applications can interface with, allowing them to indirectly access the measurements of physical sensors.
!!!!The concept of virtualising sensors has been studied in a few research works.  

!!One such work Virtual Node Layer (VNL) is a programming abstraction solution that provides a predictable virtual representation of the unpredictable physical resources of sensor nodes. The researchers in \cite{Brown:2007:VNL:1317103.1317105} propose a generic architecture that simplifies the design and implementation of application development for WSNs. The work utilizes a broad range of virtualisation schemes that help masking the uncertain nature of WSNs. For example, arbitrary regions of the WSN are identified by emulated virtual sensors (VS). This not only makes application programming easier but also increases fault tolerance and provides additional redundancy. On the down side, the clustering approach does not promote concurrency and gives lower accuracy to position sensitive applications (e.g. tracking).


The authors in \cite{1648519} describe a virtual solution to implement programming interfaces that allow user applications to aggregate sensor data through VSs. This approach allows the heterogeneous physical sensor data to be combined into an abstract measurement. For example the combination of wind speed, boom angle and load can provide a value to determine whether a construction crane has exceeded its safe working load. This abstraction provides more flexibility and generality to the application developer defining the necessary operations on the various physical data sources via a simple API. The particular focus of this project is on efficient in-network processing as well as data aggregation techniques regarding different data types. The API provided to the developers offer standard data types such as temperature, angle, location, etc. and can be further extended as required. Applications can access sensor data by delegating the corresponding operations to the VS. Using the virtual sensor approach user applications are essentially decouple  from the physical data sources so changes in the physical set-up do not effect the applications directly. 

Another VS abstraction is presented in \cite{5721776}. VSs are implemented using the SPINE2 framework \cite{5346155} that provides programming abstractions based on the task-oriented paradigm[cite]. SPINE2 uses graphical constructs to simplify the development of collaborative sensor applications in wireless body sensor networks (WBSN). The project proposes a multi-layer task model that abstracts and combines the components of the WBSN including processing and sampling tasks into a VS.
\subsection{Cloud computing aproaches}
The idea of utilizing technologies and services that are normally associated with the Cloud has attracted the attention of investigators in the computing and natural science domains \cite{Kurschl:2009:CCC:1806338.1806435,ahmed2011integrating,rolim2010cloud}. The benefits of jointly applying Cloud computing and WSN technologies have been identified in the literature. Existing researches such as \cite{liu2011opportunities} and \cite{hassan2009framework} argue that using dynamically scalable Cloud resources is key to provide compelling technical solutions to efficiently store and process -in real time- the exponentially increasing amount of data collected by WSNs. In order to realize the collaboration between the technologies, however, a number of technical challenges needs to be addressed. These challenges typically derive from communication reliability issues and from the resource constrained nature of WSNs.

Following the original IaaS, SaaS, PaaS Cloud paradigms \cite{lenk2009s} several recent researches proposed cloud-like WSN service models. Some of the most notable studies are outlined below.

\subsubsection{Software as a Service (SaaS)}
\cite{Kurschl:2009:CCC:1806338.1806435} describes a virtualisation approach that utilizes the virtually unlimited storage and processing power of the Cloud by combining it with WSNs. Based on the SaaS paradigm, the work proposes a model that moves sensor data and its processing out of the WSN to into the Cloud hence overcoming the limitations of in-network computation. The proposed model has a distributed architecture which focuses on offline processing. The parts of the system that are hosted in the Cloud are based on the pipe and filter design pattern and offers great flexibility regarding the platforms and programming languages that can be used for the implementation. This provides a number of benefits including
better integration of heterogeneous WSN platforms as well as scalability of processing power and data storage.  Additionally, data and results can be easily shared amongst multiple users and accessed globally via the Internet.

\subsubsection{Network as a Service (NaaS)}

Serviceware \cite{6529470} is a Service Oriented Architecture Based (SOA) middleware that utilizes infrastructure virtualisation techniques to reduce maintenance cost and simplify system operation. Based on the NaaS cloud paradigm, Serviceware exposes the physical WSN substrate to multiple users in the form of virtual WSNs as an on-demand service. To promote infrastructure re-usability, Serviceware specifies an infrastructure slicing approach for cloud based WSNs. Through this approach, Serviceware allows the sharing of WSN resources by implementing a middleware layer, which utilizes multi-threading, on top of a WSN OS. The limitation of this approach is that it relies on the underlying OS's multi-threading support.



\subsubsection{Sensing and Actuation as a Service (SAaaS)}
The study in \cite{SAaaS} presents the SAaaS business model and describes the implementation details of the underlying infrastructure to provide the service to the Internet of Things (IoT) world. Commoditization of generic sensing and actuation utilities was enabled by virtualising the participating device resources, offering them as leased services, in accordance with the SLA and QoS requirements. The work also looks beyond WSNs and by including other portable devices such as PDAs and mobile phones equipped with sensors and actuators, offered on a volunteer basis, as part of this infrastructure. The proposed software stack implemented the following three components: 1) The hypervisor, operating on the nodes, to virtualise and abstract the sensing and actuating resources, 2) The Volunteer Cloud Manager, operating on the Cloud, that provides the functionalities to manage the offered services and 3) The Autonomic Enforcer. operating on the Cloud, that mediates between the virtual resources and the SAaaS Clouds, and enforces the policies based on SLAs. Ultimately, the work presented an example to successfully applying virtualisation to Cloud inspired paradigms in a WSN environment, focusing on service management and delivery, but did not address the issue of concurrency.  


\subsubsection{Sensor as a Service (SenaaS)}

SenaaS, another cloud inspired model, was introduced in \cite{alam2010virtualizing}. The research presents a semantic enhanced framework that exposes the capabilities of sensor nodes and other IoT devices as Web Services (WS). It is based on event driven SOA that uses virtualisation as its key technology to provide a service virtualisation layer. Access to the physical layer is provided through interfaces that follow an adapter oriented approach. Moreover, a semantic-overlay layer was created to provide a model of the underlying resources using an ontology.  The framework successfully demonstrated a simple way to gather telemetry information by exposing sensor resources as virtual services. The work, however, did not target challenges that are associated with in-network node level operations such as programmability, adaptability or constraint/resource management. 



\subsection{Data Virtualization (DV)}
The ROADNet \cite{Rajasekar:2005:ASD:1080885.1080892} project introduces a Virtual Object Ring Buffer (VORB) framework  to virtualise sensor data. The researchers proposed a catalogue that contains meta data about the sensors' system wide properties in a descriptive and structural manner. This catalogue is then used by the users to discover and access the abstract data stream of sensors of interests. At its core, VORB virtualises data streams by using the Storage Resource Broker (SRB) data grids system, which is used to federate storage resources, from multiple ORBs. The framework offers a highly scalable data virtualisation solution due to the fact that the limitations of the underlying physical structures are not limiting factors in deploying VORB. 

Making data available as a service, \cite{yuriyama2010sensor} presents the Sensor-Cloud Infrastructure that virtualises multiple physical sensors as a virtual sensor. This allows access to sensor resources by multiple users through automatic provisioning that can scale the allocated resources dynamically. The work identifies two areas that needs to be addressed in order to deliver the proposed service. One is sensor system management and the other is sensor data management. For the management of the system and to represent the relevant characteristics of the physical nodes the Sensor Modelling Language (SensorML) was used. This allowed the mapping of physical resources to virtual ones, that in turn was made accessible via a Web user interface. Virtualisation of sensor data was achieved by implementing publisher/subscriber mechanisms which allowed multiple user applications to subscribe to one or more virtual sensors that published real-time data. On one hand, there are obvious merits to the sensor virtualisation that Sensor-Cloud Infrastructure provides. On the other hand, grouping many physical sensors into a virtual one would result in the loss of position granularity, making it unsuitable for location sensitive applications. 






%\input{middleware}

\subsection{Mobile agent approach}
Mobile agents are independent non-stationary computer programs that perform various actions autonomously on behalf of an user in a distributed environment. It can freely to move around the network and can transport its  code and state with it to another node, where it can continue execution. This ability allows them to execute asynchronously by moving to and interacting with heterogeneous hosts. Additionally, this behaviour enables mobile agents to reduce network load and overcome network latency as well as to adapt dynamically to different execution environments \cite{Lange:1999:SGR:295685.298136}.



SensorWare \cite{Boulis03designand} is a framework that was designed to create WSNs that are open to multiple transient users with dynamic needs. By following the active networks paradigm, the framework defines dynamically deployable mobile scripts to re-program WSNs with a particular focus on programming it as a whole. SensorWare, furthermore, extends the original active networks model by implementing a VM that in turn interprets the high-level scripts that are dispersed through the network as data packets. This allows individual nodes to autonomously re-task other WSN nodes via these data packets. Additionally, code replication and migration are also supported. SensorWare abstractions enable the characterisation of an execution environment that, due to the abstraction level it implements, allows mobile scripts to be very concise. Moreover, it offers an event-based, high-level programming language model that provides mechanisms to share resources and significantly simplify implementing distributed algorithms. The disadvantage of defining a model with such high-level abstractions is that, unlike other approaches such as Maté, SensorWare cannot fit into more resource constrained nodes. Maté on the other hand sacrifices ease of programming for an ultra compact instruction set, which ultimately requires more packets (program code) to be transmitted over the network. 

The study presented in \cite{lynch2009middleware} introduces a novel middleware platform architecture that supports the native execution of mobile agents on WSN nodes without any code interpretation or translation. Hence the self-contained agents can follow events of interest by transmitting themselves over the network. This offers a flexible way to reprogram WSN nodes without the need to replace the entire firmware. The proposed system was implemented in the C programming language on top of SOS \cite{han2005dynamic}, with a particular focus on minimizing the time and energy cost of transmitting the agent code. The
system provides neighbour discovery, inter-agent communication, migration, routing, and remote module fetching mechanisms.

Agilla is another mobile agent that was designed to support multiple applications, implemented on top of TinyOs, using tuple space abstraction (a form of distributed shared memory \cite{Carriero:1989:LC:63334.63337}) model. At its core, a VM kernel controls all running agents on the node using a round-robin scheduler for concurrency. The use of the round-robin scheduling algorithm is an inherent limitation of TinyOS's task based execution model. While it ensures starvation free scheduling and it is simple to implement, it does not support prioritization thus it is not suitable to provide dynamic resource management (QoS). Moreover, Agilla uses an assembly like language, that although very compact, does not provide high level abstractions and makes programming more difficult. 


\subsection{Query Processing Approach}
Research efforts have been made to abstract the WSN into a distributed Database (DB) which uses declarative query languages for data acquisition. Examples of this model include COUGAR \cite{yao2002cougar} and TinyDB \cite{madden2005tinydb}.

COUGAR was one of the first approaches to abstract away much of the
complexity of WSNs by viewing it as tables based on the database paradigm . COUGAR threats the entire WSN as a virtual database system and implements management operations using SQL like queries. In-network query processing is optimized in order to reduce resource usage and extend battery lifetime.  

TinyDB was implemented on top of TinyOS, yet it does not require the developer to write code using nesC (the native programming language for TinyOS), instead it provides an interface to extract sensor data using simple DB queries. Furthermore, it supports multi-hop communication and dynamic node reconfiguration at runtime. TinyDB also introduces an advanced energy saving scheme by using semantic routing trees which help reducing the amount of communication involved with the queries.

SwissQM \cite{mueller2007swissqm} takes data acquisition to another level by implementing a VM that runs optimized byte code instead of queries to process the telemetry data that was extracted from the WSN. Furthermore, SwissQM builds upon the data aggregation model that was first introduced in TinyDB and offers a user extensible instruction set. The work highlights and addresses the following two major limitations of existing systems: \emph{1) lack of data independence, and 2) poor integration with the higher layers of the data processing chain.}

DB approaches can simplify programming and reduce power consumption but in the same time they lack real-time application and multi platform support. Moreover, researchers have identified \cite{hadim2006middleware,yick2008wireless} that in many cases they do not provide sufficient detection accuracy of spatial and temporal relationships between events as required by hard real-time applications.

\subsection{Sensor Network Virtualisation (SNV)}
SVN has been defined \cite{islam2012survey,Chowdhury2009Network} as the separation of the roles of a traditional service provider into two distinct entities: one is sensor infrastructure provider (SInP) that manages the physical substrate; the other one is virtual network service provider (VNSP) who aggregates resources from several SInPs and offers end-to-end services to application level users. Virtual WSNs are composed by a subset of sensor nodes -connected through virtual links- that are typically allocated to particular application or task \cite{Lim:2009:VFH:1644038.1644080}.

The authors in \cite{islam2012virtualization} survey the challenges of using virtualisation in federated WSN environment. The challenges primarily derive from the heterogeneity of multi vendor devices as well as from the conflicting economic interests and strict administrative control imposed by the different WSN owners. Moreover, the survey identifies that by utilizing network virtualisation, heterogeneous WSNs can coexist on a shared physical infrastructures, providing flexibility, promoting diversity, ensuring security and increasing manageability.

One recent work is that attempts to overcome the aforementioned challenges is the VITRO project \cite{6076694}. VITRO defines a framework of a complete system architecture model for virtual WSNs. The work advocates solutions that can take advantage of virtualisation across the system from node-level to service-level. The report in \cite{sarakis2012framework} presents the details of the VITRO framework for efficient service provisioning in which the idea of using traditional virtualisation techniques is applied to WSNs. The framework in the project allows several virtual networks to coexist in a federated fashion, spanning across multiple administrative domains and interconnected via the Internet. The collaboration amongst a subset of sensor nodes supports new dynamically created applications and services that are beyond the scope of the original deployment. The aim of the VITRO architecture was to enable interoperable, adaptive and scalable virtual network platforms. 

The SenShare \cite{FRESnel_multi_application_WSN} project proposes the idea of open access infrastructures that can support multiple co-running applications. SenShare builds upon the concept of overlay networks and separates the applications from the underlying infrastructure. Furthermore, it provides and isolated execution environment for each application through a node level abstraction layer. By decoupling the ownership and the management of the network substrate, the efficient sharing of a pool of resources is promoted.
 
The benefits that are provided by SNV are generally identified as improved manageability and security as well as increased flexibility \cite{Chowdhury2009Network
	}.
\subsection{Conclusion}
This section is still a working progress so please ignore it!!!!!!!!!!!!!!!!!!!!!!!?????????????????????????

\bigskip
This section has reviewed several previous researches that successfully utilized various virtualisation techniques in the context of sensor networks. The review has clearly demonstrated the versatility of virtualisation and has highlighted the potential implications of applying the technology to WSNs regarding both benefits and challenges. 

The divergence of potential WSN application domains together with the severe resource limitation makes system design difficult. In the same time, virtual system run times are required to be adequately efficient and lightweight and support various user applications. 

Existing virtualisation approaches often focus on platform \cite{simon2005squawk} or application specific solutions \cite{levis2004bridging}. Over customization, however reduces re-usability and can further aggravate interoperability issues. Future WSNs are expected to be flexible, multi-purpose infrastructures to accommodate numerous concurrent users an support a wide range of versatile applications.  Moreover, existing WSN deployments need to be able to dynamically adapt to changes in the environment or in application requirements. Therefore the ability to remotely reconfigure and reprogram the individual sensor nodes or the whole network is imperative as it was demonstrated in \cite{koshy2005vmstar} and \cite{Michiels:2006:DDA:1176866.1176868} respectively. 

\begin{comment}
content...



The abstraction provided by virtualisation needs to be generic enough to simplify the integration of new functionality or components but in certain cases they have to be optimized to be efficient in specific domains.  

Virtualisation techniques at the different layers of the WSN infrastructure offer functionalities that are unique to the specific layer, however, they do not consider the other layers. 

The virtual machine approaches for example promote application portability and programming simplification, although the byte code interpretation mechanisms introduce computation overhead.

Based on the approaches that were reviewed in this section we identified a number of challenges that have not been fully addressed. This include the following areas:

i)
ii)
iii)

\end{comment}
\pagebreak
\section{Expected progress with respect to the state of the art / current literature / current practice of research}


\pagebreak
\section{Research methodology}
It is proposed to perform a rigorous study in the context of WSNs in order to meet the objectives outlined in section 2 using the following systematic approach:

\begin{itemize}
\item Conduct a comprehensive review of the relevant areas with emphasis on WSN virtualisation techniques (node level, network level) and  WSN programming.
\item Analyse the review of current literature/practice to gain insight and identify open research challenges.
\item Design and develop a new framework for WSN virtualisation by combining existing and novel techniques. 
\item Design and create appropriate experiments by setting up and configuring a virtualised WSN test bed. 
\item Evaluate the implemented framework against the desired characteristics specified in section 2.1.
\item Validate the scalability and performance of the approach and architecture against traditional approaches and analyse the implications of combining different virtualisation techniques within the various layers of the WSN infrastructure.
\item Identify the optimal configuration settings with the aim of achieving a balance between maximum resource utilization/performance and minimum power consumption/management complexity.
\end{itemize}
\pagebreak
\section{Work plan}
\include{work_plan}
\pagebreak
\section{Ethical issues}
None.






%Many different mechanisms for reprogramming sensor nodes have been developed ranging from full image replacement to virtual machines.

Dealing with heterogeneity raises severe obstacles to interoperability.

VMs and other systems software facilitates the mechanisms to programmatically control node operations at runtime. It enables 
standardizing the programming interfaces and synchronizing connectivity between the heterogeneous hardware platforms

The VMs main goals are to provide an abstraction layer from the possibly heterogeneous hardware platform of different sensor nodes and to improve robustness by running applications in a safe execution environment. 

It is the individual nodes' duty to perform the different tasks such as sensing, acting, signalling and processing. However, the real strength of WSNs lies in the ability of the motes to collaborate and aggregate data.  

To meet future requirements new WSN designs need to provide mechanisms to dynamically allocate resources and serve diverse applications together with multiple users.

WSN OS bridges the gap between hardware simplicity and application complexity, and it plays a central role in building scalable distributed applications that are efficient and reliable

to optimize properties such as network capacity and QoS. Since processing and transmission
power in nodes are the essential consumers of energy, it is
also necessary to optimize the number of hops traversed by
packets.

The advantages of virtualization over native solutions includes faster development cycle, portability, security, and reuse of design tools and libraries.

 The Virtual Sensor can exist
 either in-field as a thin layer of virtualization software that is
 executed on physical sensors or it can be a mathematical
 model for aggregating information residing in a sensor
 management platform similar to [23].

Virtualization, in this case, takes on a slightly different meaning. When we think of virtualized, or overlay networks, we think of an alternate, logical instantiation of the network. Virtualized networks are one key application of SDN, which is has traditionally been a separation of the control and data planes.

A trend and idea to move towards the pervasive Cloud.
A limit of current Cloud implementations is the absence of mechanisms to effectively manage inputs from the physical word.



In particular we show how reusing existing assets, using middleware, creating abstract interfaces and decoupling the components lead to a more scalable and flexible software system. The benefits of the implemented solution regarding the scalability and flexibility
of our robotic systems are exposed in the fourth section

The third and final conceptual group comprises those middleware architectures that focus on modularity and service orientation.
Modularity allows for more flexibility in the software development process, and the definition of services makes it possible to reduce the software features deployed to those that are actually required by the application, thus saving resources on the sensor nodes. The two concepts complement each other, as services are commonly implemented in the form of interchangeable modules. 




As I see it, a well-defined interface is one where the actions that will be performed by the service provider are clear and unambiguous.


\subsection{Virtualization (VMs)} 

VM supports dynamic application loading for TinyOS based Wireless Sensor Networks.Support of code propagation, app-level communication and result aggregation back to main node. 

VMs

\begin{itemize}
	\item Scylla
	\item Mate
	\item Bombilla
	\item Melete - Supporting concurrent applications in wireless sensor networks
	\item Jelatine
	\item ASVM (Extended Mate)
	\item SensorWare - Design and implementation of a framework for efficient and programmable sensor networks
	\item DVM
	\item DAViM
	\item QM
	\item VMSTAR
	\item TinyReef
	\item SwissQM
	\item
	\item Insense VM
	\item Darjeeling
	\item TakaTuka
	\item Squawk
	\item Sentilla ??
\end{itemize}

Middleware

\begin{itemize}
	\item Impala
	\item Agilla -  a mobile agent middleware for self-adaptive wireless sensor networks
	\item Servilla - a flexible service 	provisioning middleware for heterogeneous sensor networks
\end{itemize}


DB

\begin{itemize}
	\item COUGAR: the network is the database
	\item TinyDB
	
	
\end{itemize}



WSN OS

\begin{itemize}
	\item Contiki
	\item TinyOS
	\item Nano-RK
	\item LiteOS
	\item MANTIS
	\item SOS
	\item NucleOS
	\item InceOS
	\item SenSmart
	\item Lorien
\end{itemize}

Virtual Sensor

\begin{itemize}
	\item SenQ
\end{itemize}

Service Virtualisation

\begin{itemize}
	\item SenaaS
	\item SAaaS
\end{itemize}

Virtual Networks

\begin{itemize}
	\item VITRO
\end{itemize}

\begin{itemize}
	\item code mobility
	\item virtualised processor architecture
	\item power management
	\item inter device communication
	\item error recovery
	\item cost of on the fly compilation
	\item memory footprint (code and data)
	\item execution model
	\item availability
	\item remarkable characteristics
	\item application execution method
	\item application or domain specific
	\item 
\end{itemize}	
	
\subsection{WSN CONSTRAINTS} 

\begin{itemize}
\item difficult to access
\item need for physical access to reprogram 
\item limited control and manageability
\item fundamentally unpredictable 

\item prone to interference from other more powerful radios
\item physical obstructions can limit the communication range
\item unreliable data distribution / communication

\item incompatibility due to heterogeneity of devices
\item lack of simple high-level programming APIs

\item limited CPU, RAM, STORAGE, SIGNAL RANGE
\item limited power source
\end{itemize}

\subsection{WSN REQUIREMENTS}

Pablo:
• dynamic data flows
• concurrent data flows
• policy-based flows
• demand-based flows
• scalability
• QoS support
• consistency (responsiveness to network changes)
• adaptability (fault tolerance, recoverability) -> this largely same as above
• reliable control paths
• small overhead compared to distributed solutions
• constraint management
• mobility support

Roland:
• concurrency
• abstraction of heterogeneity
• high level programming abstractions
• application isolation
• local dynamic adaptability to environment
• distributed adaptability
• re-programmability
• application resource management (reservation and guarantees)
• constraint management
• application/vm/service mobility and migration


\begin{itemize}
\item efficient energy conservation
\item power harvesting capability
\item low-cost
\item small size
\item reasonable communication range

\item sufficient storage capacity
\item sufficient RAM capacity
\item CPU performance
\item various sensing and acting capability


\item flexible deploy-ability (random vs organized)
\item resistance against harsh environmental conditions
\item reliability
\item predictability
\item pro-activeness
\item responsiveness
\item self-adaptability
\item scalability
\item QoS
\item security
\item fault tolerance
\item reprogram-ability 
\item self-configuration
\item unattended operation
\item cooperative network behaviour 
\end{itemize}


Finally
,
w
e
m
ust
ac
kno
wledge
the
prev
alence
of
message
loss
due
to
wireless
in
terference
or
noise

 standard overlays falter
 as a deployment path for radical architectural innovations in
 at least two ways. First, overlays have largely been in use as
 means to deploy narrow fixes to specific problems without any
 holistic view. Second, most overlays have been designed in the
 application layer on top of IP; hence, they cannot go beyond
 the inherent limitations of the existing Internet

The neighbourhood is still not a programming primitive in the sensor network community

symmetric vs asymmetric multiprocessing
pipelined SIMD (distributed memory, shared memory)
HAL is provided by the OS
OS uses device drivers to communicate with HW



% % % % % % % % % % % % % % % % % % % % % % % % % % % % % % % % % % % % % % % % % % % 




\pagebreak

% This includes all references from the BibTeX file in the bibliography
%\nocite{*}

\begin{spacing}{1}
\bibliographystyle{IEEEtran}
\bibliography{bibliography}
\end{spacing}

\end{document}

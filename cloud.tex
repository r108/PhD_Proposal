\subsection{Cloud computing aproaches}
The idea of utilizing technologies and services that are normally associated with the Cloud has attracted the attention of investigators in the computing and natural science domains \cite{Kurschl:2009:CCC:1806338.1806435,ahmed2011integrating,rolim2010cloud}. Efforts to overcome the new challenges that implementing Cloud concepts into WSNs imposes have increased over the past years from both academia and industry. The benefits of jointly applying Cloud computing and WSN technologies have been identified in the literature. Existing researches such as \cite{liu2011opportunities} and \cite{hassan2009framework} argue that using dynamically scalable Cloud resources is key to provide compelling technical solutions to efficiently store and process -in real time- the exponentially increasing amount of data collected by WSNs. In order to realize the collaboration between the technologies, however, a number of technical challenges needs to be addressed.


\subsubsection{Software as a Service (SaaS)}
\cite{Kurschl:2009:CCC:1806338.1806435} describes a virtualisation approach that combines computing resources in the Cloud with WSN to complement each other. By utilizing virtually unlimited storage and processing power, Cloud computing opens up opportunities for new business models.  Following the SaaS paradigm, the work proposes a model that moves sensor data and its processing out of the WSN to into the Cloud hence countervail the limitations of in network computation. The proposed model has a distributed architecture which focuses on offline processing. The parts of the system that are hosted in the Cloud are based on the pipe and filter design patter and offers great flexibility regarding the platforms and programming languages that can be used for the implementation. This provides a number of benefits including
better integration of heterogeneous WSN platforms as well as scalability of processing power and data storage.  Additionally, data and results can be easily shared amongst multiple users and accessed globally via the Internet.

\subsubsection{Network as a Service (NaaS)}

Serviceware \cite{6529470} is a Service Oriented Architecture Based (SOA) middleware that utilizes infrastructure virtualisation techniques to reduce maintenance cost and simplify system operation. Based on the NaaS cloud paradigm, Serviceware exposes the physical WSN substrate to multiple users in the form of virtual WSNs as an on-demand service. To promote infrastructure re-usability, Serviceware specifies an infrastructure slicing approach for cloud based WSNs. Through this approach, Serviceware allows the sharing of WSN resources by implementing a middleware layer -that utilizes multi-threading- on top of a WSN Operating System. The limitation of this approach is that it relies on the underlying OS's multi-threading support.



\subsubsection{Sensing and Actuation as a Service (SAaaS)}

In \cite{SAaaS} the authors introduced a cloud-like paradigm to the Internet of Things (IoT) world. The study presents the SAaaS business model and describes the implementation details of the underlying infrastructure to provide such service. Commoditization of generic sensing and actuation utilities was enabled by virtualising the participating device resources, offering them as leased services, in accordance with the SLA and QoS requirements. The work also looks beyond WSNs and by including other portable devices such as PDAs and mobile phones equipped with sensors and actuators - offered on a volunteer basis- as part of this infrastructure. The approach taken by the developers was to customize and combine existing solutions wherever available or create new ones where necessary.  The proposed software stack implemented the following three components: 1) The hypervisor -operating on the nodes- to virtualise and abstract the sensing and actuating resources; 2) The Volunteer Cloud Manager -operating on the Cloud- that provides the functionalities to manage the offered services; 3) The Autonomic Enforcer -operating on the Cloud- that mediates between the virtual resources and the SAaaS Clouds, and enforces the policies based on SLAs. Ultimately, the work presented an example to successfully applying virtualisation to Cloud inspired paradigms in a WSN environment, focusing on service management and delivery, but did not address the issue of concurrency. Moreover, the services were made accessible via Web interfaces which lacks flexibility in terms of its potential use. 


\subsubsection{Sensor as a Service (SenaaS)}

SenaaS, another cloud inspired model, was introduced in \cite{alam2010virtualizing}. The research presents a semantic enhanced framework that exposes the capabilities of sensor nodes and other IoT devices as Web services. It is based on event driven, service oriented architecture (SOA) that uses virtualisation as its key technology to provide a service virtualisation layer. Access to the physical layer is provided through interfaces that follow an adapter oriented approach. Moreover, a semantic overlay layer was created to provide a model of the underlying resources using ontology.  While the framework successfully demonstrated a simple way to gather telemetry information by exposing sensor resources as virtual services. The work did not target challenges that are associated with in-network node level operations such as programmability, adaptability or constraint/resource management. 


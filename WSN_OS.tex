\section{Operating Systems for Wireless Sensor Networks}
The role of an Operating System is to provide an interface between user applications and the underlying hardware resources \cite{Stallings:1998:OSI:272982}. Memory and CPU management, process scheduling for multitasking are some of the main responsibilities of Operating Systems.
Due to the unique characteristics of WSN nodes Operating System design approaches for WSNs greatly deviate from traditional Operating Systems, hence the realization of the basic OS services is a non-trivial problem. This section summarizes two rather comprehensive survey that discuss the challenges of Operating Systems design for Wireless Sensor Networks and provides a comparison amongst different existing WSN Operating Systems.
\cite{Reddy:2009:WSN:1593545.1593549} presents a classification framework that analyses the different OS architectures that are used for WSN OS design. 
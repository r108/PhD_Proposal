\subsection{Sensor Network Virtualisation (SNV)}
SVN has been defined \cite{islam2012survey,Chowdhury2009Network} as the separation of the roles of a traditional service provider into two distinct entities: one is sensor infrastructure provider (SInP) that manages the physical substrate; the other one is virtual network service provider (VNSP) who aggregates resources from several SInPs and offers end-to-end services to application level users. Virtual WSNs are composed by a subset of sensor nodes -connected through virtual links- that are typically allocated to particular application or task \cite{Lim:2009:VFH:1644038.1644080}.

The authors in \cite{islam2012virtualization} survey the challenges of using virtualisation in federated WSN environment. The challenges primarily derive from the heterogeneity of multi vendor devices as well as from the conflicting economic interests and strict administrative control imposed by the different WSN owners. Moreover, the survey identifies that by utilizing network virtualisation, heterogeneous WSNs can coexist on a shared physical infrastructures, providing flexibility, promoting diversity, ensuring security and increasing manageability.

One recent work is that attempts to overcome the aforementioned challenges is the VITRO project \cite{6076694}. VITRO defines a framework of a complete system architecture model for virtual WSNs. The work advocates solutions that can take advantage of virtualisation across the system from node-level to service-level. The report in \cite{sarakis2012framework} presents the details of the VITRO framework for efficient service provisioning in which the idea of using traditional virtualisation techniques is applied to WSNs. The framework in the project allows several virtual networks to coexist in a federated fashion, spanning across multiple administrative domains and interconnected via the Internet. The collaboration amongst a subset of sensor nodes supports new dynamically created applications and services that are beyond the scope of the original deployment. The aim of the VITRO architecture was to enable interoperable, adaptive and scalable virtual network platforms. 

The SenShare \cite{FRESnel_multi_application_WSN} project proposes the idea of open access infrastructures that can support multiple co-running applications. SenShare builds upon the concept of overlay networks and separates the applications from the underlying infrastructure. Furthermore, it provides and isolated execution environment for each application through a node level abstraction layer. By decoupling the ownership and the management of the network substrate, the efficient sharing of a pool of resources is promoted.
 
The benefits that are provided by SNV are generally identified as improved manageability and security as well as increased flexibility \cite{Chowdhury2009Network}.
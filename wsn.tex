\section{Wireless Sensor Networks and their Characteristics}
Wireless Sensor Networks are distributed systems made up of numerous tiny embedded devices, called motes, that interact with the physical world. The idea behind using these small devices is to enable the user to collect telemetry data from the environment that they are deployed in using various sensors, and also to react to changes using on-board actuators. Furthermore, the motes can perform calculations and support the wireless infrastructure that each member of the WSN uses for communication.  The devices are equipped with integrated radio transmitters which they use to communicate with neighbouring devices within proximity. Motes typically form a mesh connection with one another that enables them to relay data over distances that exceeds the range of any single device. Although, any individual mote can perform various independent tasks, their real strength lies in their ability to communicate, and thus, to perform more complex tasks as a collective whole. Devices in Wireless Sensor Networks are typically powered by batteries and intended to operate for a long period of time. Over the years, several power-saving mechanisms have been developed to reduce the power consumption of the nodes and as a result prolong their operation lifetime. While power consumption is one of the major constraints, there are a number of other limitations that are associated with WSNs such as low memory and computational capability as well as the lack of a simple way to maintain and reprogram them after deployment. Traditionally, wireless sensor networks consist of homogeneous sensor motes which are programmed for some very specific purpose \cite{ammari2013art,mittal2012contemporary}. 
Sensor nodes can typically be characterized by their physical traits such as power consumption, communication range, memory and storage capacity, computational capability, etc.
Additionally, based on their software functions, WSN nodes can be divided into two categories. One is node-level operations that involves computational tasks and hardware control such as the radio, RAM, CPU and sensors. The other one is network-level operations which consists of communication related tasks such as routing. For a more detailed and concise
classification of wireless sensor nodes, the reader can refer to the following publications: \cite{davis2012survey,mottola2011programming,Akyildiz2002393}.
In the next sections the aforementioned device constraints and design challenges are further explored through a number of case studies.


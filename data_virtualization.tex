\subsection{Data Virtualization (DV)}
Data virtualisation refers to the encapsulation and/or aggregation of data from disparate sources and providing a single point where it can be viewed from. DV is applied, by providing an intermediate layer 
that hides the technical characteristics of how and where data is stored from the applications that want to access it \cite{book:1035720}. 

The ROADNet \cite{Rajasekar:2005:ASD:1080885.1080892} project introduces the Virtual Object Ring Buffer (VORB) framework  to virtualise sensor data. The researchers propose a catalogue that contains metadata about the sensors' system wide properties. This catalogue is then used by the users to discover and access the abstract data stream of sensors of interests. At its core, VORB virtualises data streams by using the Storage Resource Broker (SRB) \cite{baru1998sdsc} data grids system, which is used to federate storage resources, from multiple Object Ring Buffers (ORBs) \cite{moore2004data}. The framework offers a highly scalable data virtualisation solution due to the fact that the limitations of the underlying physical structures are not limiting factors in deploying VORB. 

The work in \cite{yuriyama2010sensor} presents the Sensor-Cloud Infrastructure that virtualises multiple physical sensors as a virtual sensor and makes data available as a service. This allows access to sensor resources by multiple users through automatic provisioning that can scale the allocated resources dynamically. The work identifies two areas that needs to be addressed in order to deliver the proposed service. One is sensor system management and the other is sensor data management. For the management of the system and to represent the relevant characteristics of the physical nodes the Sensor Modelling Language (SensorML) \cite{botts2007opengis} was used. This allowed the mapping of physical resources to virtual ones, that in turn was made accessible via a Web user interface. Virtualisation of sensor data was achieved by implementing a publish/subscribe \cite{book:1001628} mechanisms which allowed multiple user applications to subscribe to one or more virtual sensors that published real-time data. On one hand, there are merits to the sensor virtualisation that Sensor-Cloud Infrastructure provides including improved resource utilization and simplified management. On the other hand, grouping many physical sensors into a virtual one would result in the loss of position granularity, making it unsuitable for location sensitive applications. 






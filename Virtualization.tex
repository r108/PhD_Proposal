\subsection{Virtualisation}
The notion of virtualisation has been around for decades, since the debut of virtual memory \cite{denning1970virtual} and virtual computer systems \cite{goldberg1973architectural}.
This section provides a brief introduction to virtualisation and discusses the basic characteristics of the technology.

The concept that provides one or more functional environments, which are independent from the underlying resources, is known as virtualisation. At its core, virtualisation simply refers to the decoupling of computation — processing, sensing/actuating, storage and networking — from the resource that performs the action. It is used to conceal the actual physical traits of computing resources in order to simplify the methods through which those resources can be interacted with. The technology exists in many different forms, depending on the degree of abstraction it provides at the hardware, instruction set, system, process, library or application level. 

%Depending on the desired goal the virtualisation process may divide a single computing resource into multiple virtual copies to advocate concurrency or may combine several independent resource units into one virtual entity to increase computational capacity and supply redundancy. 

One form of virtualisation facilitates multiple application instances to concurrently execute in a virtual environment. Another form allows the aggregation of hardware resources that are physically distributed across multiple devices into an individual virtual unit. The benefits of this is that it offers load balancing, redundancy and abstraction of heterogeneous resources. Such abstractions facilitate the translation of resources into a generic virtual resource that in turn can be consumed by various user applications via high-level software interfaces. 


Virtualisation is typically implemented by applying various techniques such as: hardware/software partitioning, aggregation or composition, time-sharing and indirection or encapsulation \cite{fortes2005guest} which are further outlined below.

\paragraph{Partitioning:} Facilitates the sharing of a single physical resource amongst multiple users or application instances by dividing it into several logical portions (i.e. hard-partitioning). Through multiplexing, many concurrent applications can consume an individual resource by taking turns to access it(i.e. time-sharing).   

\paragraph{Aggregation:} A technique that makes multiple, typically identical, physical units appear to function as a unified logical unit. Thus, the resulting resource pool can be viewed as a single virtual element.

\paragraph{Composition:} A mechanism to combine a number of simpler, typically different, resources into a more complex virtual resource that in turn offers additional functions not provided by any of the original components. 

%Resource allocation and hardware arbitration are the main enablers of virtualisation, through which the substrate physical computing assets %are delineated as software abstractions.  Software isolation for fault mitigation and performance management are also fundamental properties %of virtualisation.

\begin{comment}

\paragraph{Application virtualisation (AV)} is achieved by encapsulating (a.k.a sandboxing) the individual instances of an application into a container from the system runtime environment that they are executed on. Relevant system files and programming libraries are also captured in this container, making applications easily portable across different platforms \cite{lunsford2009virtualization}. 

\paragraph{Network virtualisation (NV)} enables various heterogeneous network  models to co-exist on the same physical infrastructure. NV is defined by the decoupling of the
roles of the traditional Internet Service Providers (ISPs) into
two independent entities [2], [3]: infrastructure providers, who
manage the physical infrastructure, and service providers, who
create virtual networks by aggregating resources from multiple infrastructure providers and offer end-to-end services 
Although the term is rather ambiguous as many different implementations of it can be found, they all share the common objective of logically representing network resources in software. The benefits that are provided by NV include improved manageability and security as well as increased flexibility \cite{Chowdhury2009Network}.

\paragraph{Service virtualisation (SV)} Manifesting itself in various forms, service virtualisation utilises the above definitions. By integrating all necessary elements, it delivers services to user applications over the network. 

Service virtualization is used to simulate the behavior of components in an application so you can perform an accurate and timely test in a world of complex interrelated applications. Production services that may not be available for integration testing can be virtualized so the testing can take place at an appropriate time in the software development process

content...
\end{comment}

\bigskip
Virtualisation technology has now matured to the point where it can be applied to WSNs. By abstracting the hardware specific details of WSN nodes with different architectures and hiding the actual physical environment behind a generic virtual representation, heterogeneous device resources can be automatically mapped to user applications. Virtualisation can offer numerous benefits, by providing user applications with a logical view of the system, including:

\begin{itemize}
	\item Simplified application development
	\item Improved resource utilization
	\item Simplified network management
	\item Increased fault tolerance and enhanced security through isolation
	\item Greater flexibility, in the shape of application compatibility and portability
	
\end{itemize}
 

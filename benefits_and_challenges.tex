\section{Virtualization in WSNs}
Today's WSN platforms have greatly benefited from the advances in manufacturing technologies.  Motes are becoming smaller, cheaper, more efficient and capable in terms of their processing power. Despite the substantial improvements in hardware there are still significant limitations associated with application development for WSNs. The literature has identified a significant gap \cite{Fok:2009:AMA:1552297.1552299,Costa:2007:VMA:1304610.1306527,1621014,Levis:2002:MTV:635506.605407} which exists in between the complexity of the required applications and the lack of sufficient low-level hardware abstractions that would simplify the programming and management of heterogeneous sensor nodes. 

Typically, the Operating System (OS) has the central role of providing Application Programming Interfaces (API) for user applications to the underlying hardware resources \cite{hailperin2007operating,Stallings:1998:OSI:272982}. Memory and CPU management, low-level resource abstraction and task scheduling for multitasking are some of the main responsibilities of a typical OS. Over the years, important developments have been made to provide light-weight operating systems for WSNs that meet the requirements of diverse and demanding user applications. \cite{1367266,1630599,bhatti2005mantis,eswaran2005nano,s110605900,Reddy:2009:WSN:1593545.1593549}.
Due to the unique characteristics of the motes, OS design approaches for WSNs greatly differ from traditional Operating Systems, hence the realization of the basic OS services is a non-trivial problem. A classification framework, which analyses the different approaches that are used for WSN OS design, is presented in \cite{Reddy:2009:WSN:1593545.1593549}. The survey classifies WSN OSs based on their architecture, execution model, reprogramming ability, scheduling and power management. 

In today's WSN deployments application programming involves working close to the OS and solving low-level system problems that requires technical knowledge which is rarely found amongst application programmers. Hence, the need for suitable high-level programming interfaces is apparent an was identified by \cite{mottola2011programming}. The work classifies the various programming approaches  that have gained popularity in WSNs, and provides a comprehensive comparison amongst the different schemes and highlights the most important design concerns to programming WSN nodes. 

Simplifying application development is arguably the main advantage of accessing hardware resources via high level APIs. In WSNs these are normally provided in a number of different ways. Most commonly an operating system is used for hiding operational complexities from user application logic. Apart from OSs, other approaches such as virtualisation with the use of Virtual Machines (VM) \cite{Levis:2002:MTV:635506.605407,4300022,simon2005squawk,hong2009tinyvm} and other middleware \cite{Fok:2009:AMA:1552297.1552299,1621014,6529470,6671886} have been widely proposed.



VM implementation approaches can greatly differ, as it is the case between Squawk \cite{simon2005squawk}, which is a Java micro edition Virtual Machine (VM) targeted for resource constrained devices implemented on bare metal, and Mate \cite{Levis:2002:MTV:635506.605407}, a byte code interpreter that runs on top of TinyOS. Squawk's core, unlike most VMs, was mainly written in Java that makes it easily portable. It implements an application isolation mechanism that allows running multiple application instances at once and provides an API to enable developers to write wireless applications for WSN. 
Mate, on the other hand, runs as a single component of TinyOS.  It specifies 24 instructions, 8 of which can be customized for a particular application. Each instruction is only one byte long and fits into a packet that can be forwarded through the network allowing the motes to be dynamically reprogrammed. More complex programs can easily be composed of multiple packets. The work identifies the need for re-tasking as environmental conditions change and borrows ideas from Active Networks architecture in which the execution codes are transmitted via active packets. Concurrency is achieved through three separate execution contexts and data communication among contexts is made possible by a shared variable, a one word heap. Further popular virtual machines include \cite{hong2009tinyvm}, another byte code interpreter, and \cite{Barr:2002:NSS:509526.509528} which describes a distributed operating system implemented as a single system image of a unified Java virtual machine installed across the nodes that comprise an ad-hoc network.

%\href{run:/home/roland/Dropbox/_PhD/VWSN_Summary/RK_Summary.pdf}{runit}

The idea of utilizing technologies and services that are normally associated with the Cloud have been attracted the attention of WSN investigators, as it has been indicated by increasing numbers of proposed researches. One recent example is the VITRO project \cite{6076694} that specifies a framework of a complete system architecture model for virtual WSNs. The work advocates solutions that can take advantage on virtualization across the system from node-level to service-level.
In \cite{SAaaS} the authors introduce another cloud-like paradigm to the Internet of Things world. The study presents Sensing and Actuation as a Service $\left(SAaaS\right)$ through describing a model for the proposed implementation of the underlying infrastructure. Further efforts in bringing cloud concepts to WSNs have increased over the past years from academy and industry alike. The report in \cite{sarakis2012framework} presents a framework for efficient service provisioning in which the idea of using traditional techniques is applied to wireless sensor networks. Virtual Node Layer (VNL) \cite{Brown:2007:VNL:1317103.1317105} is another alternative programming abstraction solution that provides a predictable virtual representation of the unpredictable physical resources of sensor nodes. The researchers propose a generic architecture that simplifies the design and implementation of a broad range of virtualisation schemes. Moreover, \cite{6529470} presents a service based infrastructure slicing approach for virtualisation in embedded networked devices that supports cloud based WSNs.
\subsection{Mobile agent approach}
Mobile agents are independent non-stationary computer programs that perform various actions autonomously on behalf of an user in a distributed environment. It can freely to move around the network and can transport its  code and state with it to another node, where it can continue execution. This ability allows them to execute asynchronously by moving to and interacting with heterogeneous hosts. Additionally, this behaviour enables mobile agents to reduce network load and overcome network latency as well as to adapt dynamically to different execution environments \cite{Lange:1999:SGR:295685.298136}.



SensorWare \cite{Boulis03designand} is a framework that was designed to create WSNs that are open to multiple transient users with dynamic needs. By following the active networks paradigm, the framework defines dynamically deployable mobile scripts to re-program WSNs with a particular focus on programming it as a whole. SensorWare, furthermore, extends the original active networks model by implementing a VM that in turn interprets the high-level scripts that are dispersed through the network as data packets. This allows individual nodes to autonomously re-task other WSN nodes via these data packets. Additionally, code replication and migration are also supported. SensorWare abstractions enable the characterisation of an execution environment that, due to the abstraction level it implements, allows mobile scripts to be very concise. Moreover, it offers an event-based, high-level programming language model that provides mechanisms to share resources and significantly simplify implementing distributed algorithms. The disadvantage of defining a model with such high-level abstractions is that, unlike other approaches such as Maté, SensorWare cannot fit into more resource constrained nodes. Maté on the other hand sacrifices ease of programming for an ultra compact instruction set, which ultimately requires more packets (program code) to be transmitted over the network. 

The study presented in \cite{lynch2009middleware} introduces a novel middleware platform architecture that supports the native execution of mobile agents on WSN nodes without any code interpretation or translation. Hence the self-contained agents can follow events of interest by transmitting themselves over the network. This offers a flexible way to reprogram WSN nodes without the need to replace the entire firmware. The proposed system was implemented in the C programming language on top of SOS \cite{han2005dynamic}, with a particular focus on minimizing the time and energy cost of transmitting the agent code. The
system provides neighbour discovery, inter-agent communication, migration, routing, and remote module fetching mechanisms.

Agilla is another mobile agent that was designed to support multiple applications, implemented on top of TinyOs, using tuple space abstraction (a form of distributed shared memory \cite{Carriero:1989:LC:63334.63337}) model. At its core, a VM kernel controls all running agents on the node using a round-robin scheduler for concurrency. The use of the round-robin scheduling algorithm is an inherent limitation of TinyOS's task based execution model. While it ensures starvation free scheduling and it is simple to implement, it does not support prioritization thus it is not suitable to provide dynamic resource management (QoS). Moreover, Agilla uses an assembly like language, that although very compact, does not provide high level abstractions and makes programming more difficult. 


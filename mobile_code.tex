\subsection{Mobile agent approach}
SensorWare \cite{Boulis03designand} is a framework that was designed to create WSNs that are open to multiple transient users with dynamic needs. By following the active networks paradigm, the framework defines dynamically deployable mobile scripts to re-task WSNs with a particular focus on programming it as a whole. SensorWare, furthermore, extends the original active networks model by implementing a VM that in turn interprets the high-level scripts that are dispersed through the network as data packets. This allows individual nodes to autonomously re-task other WSN nodes via these data packets. Additionally, code replication and migration is also supported. SensorWare characterises an execution environment that due to the abstraction level it implements allows mobile scripts to be very concise. Moreover, it offers an event-based, high-level programming language model that provides ways to share resources and significantly simplifies implementing distributed algorithms. The disadvantage of defining a model with such high-level abstractions is that, unlike other approaches such as Maté, SensorWare cannot fit into more resource constrained nodes. Maté on the other hand sacrifices ease of programming for an ultra compact instruction set, which ultimately requires more packets (program code) to be transmitted over the network. 

Agilla is another mobile agent that was designed to support multiple applications, implemented on top of TinyOs, using tuple space abstraction (a form of distributed shared memory \cite{Carriero:1989:LC:63334.63337}) model. At its core, a VM kernel controls all running agents on the node using a round-robin scheduler for concurrency. The use of the round-robin scheduling algorithm is an inherent limitation of TinyOS's task based execution model. While it ensures starvation free scheduling and it is simple to implement, it does not support prioritization thus it is not suitable to provide dynamic resource management (QoS). Moreover, Agilla uses an assembly like language, that although very compact, does not provide high level abstractions and makes programming more difficult. 


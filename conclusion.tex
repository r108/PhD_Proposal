\subsection{Conclusion}
This section is still a working progress so please ignore it!!!!!!!!!!!!!!!!!!!!!!!?????????????????????????

\bigskip
This section has reviewed several previous researches that successfully utilized various virtualisation techniques in the context of sensor networks. The review has clearly demonstrated the versatility of virtualisation and has highlighted the potential implications of applying the technology to WSNs regarding both benefits and challenges. 

The divergence of potential WSN application domains together with the severe resource limitation makes system design difficult. In the same time, virtual system run times are required to be adequately efficient and lightweight and support various user applications. 

Existing virtualisation approaches often focus on platform \cite{simon2005squawk} or application specific solutions \cite{levis2004bridging}. Over customization, however reduces re-usability and can further aggravate interoperability issues. Future WSNs are expected to be flexible, multi-purpose infrastructures to accommodate numerous concurrent users an support a wide range of versatile applications.  Moreover, existing WSN deployments need to be able to dynamically adapt to changes in the environment or in application requirements. Therefore the ability to remotely reconfigure and reprogram the individual sensor nodes or the whole network is imperative as it was demonstrated in \cite{koshy2005vmstar} and \cite{Michiels:2006:DDA:1176866.1176868} respectively. 

\begin{comment}
content...



The abstraction provided by virtualisation needs to be generic enough to simplify the integration of new functionality or components but in certain cases they have to be optimized to be efficient in specific domains.  

Virtualisation techniques at the different layers of the WSN infrastructure offer functionalities that are unique to the specific layer, however, they do not consider the other layers. 

The virtual machine approaches for example promote application portability and programming simplification, although the byte code interpretation mechanisms introduce computation overhead.

Based on the approaches that were reviewed in this section we identified a number of challenges that have not been fully addressed. This include the following areas:

i)
ii)
iii)

\end{comment}
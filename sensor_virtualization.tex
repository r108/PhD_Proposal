\subsection{Virtual Sensor (VS)}
VSs are software entities that applications can interface with, allowing them to indirectly access the measurements of physical sensors.
!!!!The concept of virtualising sensors has been studied in a few research works.  

!!One such work Virtual Node Layer (VNL) is a programming abstraction solution that provides a predictable virtual representation of the unpredictable physical resources of sensor nodes. The researchers in \cite{Brown:2007:VNL:1317103.1317105} propose a generic architecture that simplifies the design and implementation of application development for WSNs. The work utilizes a broad range of virtualisation schemes that help masking the uncertain nature of WSNs. For example, arbitrary regions of the WSN are identified by emulated virtual sensors (VS). This not only makes application programming easier but also increases fault tolerance and provides additional redundancy. On the down side, the clustering approach does not promote concurrency and gives lower accuracy to position sensitive applications (e.g. tracking).


The authors in \cite{1648519} describe a virtual solution to implement programming interfaces that allow user applications to aggregate sensor data through VSs. This approach allows the heterogeneous physical sensor data to be combined into an abstract measurement. For example the combination of wind speed, boom angle and load can provide a value to determine whether a construction crane has exceeded its safe working load. This abstraction provides more flexibility and generality to the application developer defining the necessary operations on the various physical data sources via a simple API. The particular focus of this project is on efficient in-network processing as well as data aggregation techniques regarding different data types. The API provided to the developers offer standard data types such as temperature, angle, location, etc. and can be further extended as required. Applications can access sensor data by delegating the corresponding operations to the VS. Using the virtual sensor approach user applications are essentially decouple  from the physical data sources so changes in the physical set-up do not effect the applications directly. 

Another VS abstraction is presented in \cite{5721776}. VSs are implemented using the SPINE2 framework \cite{5346155} that provides programming abstractions based on the task-oriented paradigm[cite]. SPINE2 uses graphical constructs to simplify the development of collaborative sensor applications in wireless body sensor networks (WBSN). The project proposes a multi-layer task model that abstracts and combines the components of the WBSN including processing and sampling tasks into a VS.
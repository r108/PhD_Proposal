\subsection{Introduction}

\begin{comment}
The study of Wireless Sensor Networks (WSNs) has increasingly attracted the interest of many researchers in recent years. Their wide range of possible applications have been demonstrated in many areas. These include human habitat, \cite{mainwaring2002wireless}, health \cite{virone2006advanced} and environmental monitoring \cite{4379685,cardell2004field} as well as industry control \cite{1631480} and military surveillance \cite{6268958}.  Advances in the ever-developing field of Micro-Electro-Mechanical Systems (MEMS) in parallel with the latest developments in software technologies have created additional
opportunities to integrate WSNs into our everyday life.  As the field of WSNs matures, ,draws growing attention from industry and academy alike.
%
%\begin{comment}
The explosion of technological innovations in the past decades has irreversibly changed our way of life. Advances in the ever-developing field of Micro-Electro-Mechanical Systems (MEMS), in parallel with the latest developments in software technologies, have created new opportunities to assist our life. By adding computational intelligence to everyday objects they become smarter and more interactive. Together with other embedded devices, such as mobile phones and security cameras, Wireless Sensor Networks (WSNs) contributed to the creation of an additional sensing / actuating layer to our surroundings. This fusion of the various embedded devices into people's living environments is motivated by potential benefits such as simplified management and improved utilization of scarce resources.  

\end{comment}

In recent years, the study of WSNs has attracted the interest of many researchers from industry and academia alike.  The wide range of practical applications of the technology have been demonstrated in many areas. These include human habitat, \cite{mainwaring2002wireless}, health \cite{virone2006advanced} and environmental monitoring \cite{4379685,cardell2004field} as well as industrial control \cite{1631480} and military surveillance \cite{6268958}. Traditionally, WSNs consist of numerous homogeneous devices, that are programmed for a specific purpose or application \cite{ammari2013art,mittal2012contemporary}, and owned and controlled by a single organization \cite{Yick:2008:WSN:1389582.1389832,obaidat2014principles}. WALL GARDEN!!

%

%WSN nodes can typically be characterized by their physical traits such as power consumption, communication range, memory and %storage capacity and computational capability.
%Additionally, based on their software functions, the nodes perform two different types of operations. One is node-level operations %that involve computational tasks and hardware control such as the radio, RAM, CPU and on-board sensors/actuators. The other one is %network-level operations which include communication related tasks such as routing. A more detailed and concise
%classification of wireless sensor nodes is provided in the following publications: %\cite{davis2012survey,mottola2011programming,Akyildiz2002393}.  

Investigators in the field have recognized that the traditional WSN deployment model inherently lacks flexibility and does not facilitate efficiency in resource utilization \cite{islam2012virtualization}. In order to address these shortcomings many different research efforts have been carried out  \cite{virtual_sensor_networks,Leontiadis:2012:STS:2187181.2187188,Fok:2009:AMA:1552297.1552299}, which have focused on developing efficient techniques for creating and managing customizable and adaptable WSNs that can accommodate multiple concurrent applications running on heterogeneous nodes.  

!!Virtualisation has been proposed by the WSN research community as a promising solution to address WSN design issues \cite{islam2012survey} to achieve concurrency, improved efficiency and simplified network and service management.
Numerous previous studies have shown how WSNs can benefit from the advantages associated with virtualisation \cite{6076694} including better resource utilization and scalability, ease of service provisioning and migration as well as improved security by isolation. Moreover, virtualisation offers additional benefits through the abstraction of the underlying hardware and processor architectures. By hiding the heterogeneity of device resource and infrastructure details, and by providing high level programming interfaces to programmers, application development can be simplified and accelerated. Consequently, considering the highly distributed quality of WSNs, a generic virtual system representation may bring further benefits, allowing applications to run transparently over disparate physical processor architectures.  

Many viable virtualisation solutions have gained recognition in the general networking \cite{6272301} and computing \cite{6008687} domains. More generally the concept of virtualisation is being applied across virtually all sections of a typical IT infrastructure \cite{murphy2008virtualization}. However, the resource constrained nature of WSNs poses unique challenges that are normally not present in other domains. One of WSNs key characteristic is that the sensor nodes operate from mainly non-renewable battery power. Therefore, extensive research efforts have focused on energy-efficient virtualisation mechanisms to extend battery life, and in turn, prolong operation lifetime \cite{Hong:2012:TEE:2388871.2388872}.  While power consumption is one of the major constraints, there are a number of other limiting factors that are related to WSNs and must be considered when developing a new application.  These limiting factors typically include low memory and computational capability as well as the lack of a simple way to maintain and reprogram sensor nodes after deployment. 

Over the past years several works have been conducted as an attempt to overcome these difficulties and provide effective virtualisation alternatives to traditional WSN deployments. This section presents a summary of the various approaches to virtualisation in WSNs. The predominant focus of this review is to survey existing research works that have demonstrated how WSNs can leverage from the different virtualisation techniques that have been adopted recently. 


 